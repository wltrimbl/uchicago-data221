\documentclass[12pt]{book}

%These tell TeX which packages to use.
\usepackage{array,epsfig}
\usepackage{amsmath}
\usepackage{amsfonts}
\usepackage{amssymb}
\usepackage{amsxtra}
\usepackage{amsthm}
\usepackage{mathrsfs}
\usepackage{color}
\usepackage{eurosym}
\usepackage{times}
%Here I define some theorem styles and shortcut commands for symbols I use often
\theoremstyle{definition}
\newtheorem{defn}{Definition}
\newtheorem{thm}{Theorem}
\newtheorem{cor}{Corollary}
\newtheorem*{rmk}{Remark}
\newtheorem{lem}{Lemma}
\newtheorem*{joke}{Joke}
\newtheorem{ex}{Example}
\newtheorem*{soln}{Solution}
\newtheorem{prop}{Proposition}

\newcommand{\lra}{\longrightarrow}
\newcommand{\ra}{\rightarrow}
\newcommand{\surj}{\twoheadrightarrow}
\newcommand{\graph}{\mathrm{graph}}
\newcommand{\bb}[1]{\mathbb{#1}}
\newcommand{\Z}{\bb{Z}}
\newcommand{\Q}{\bb{Q}}
\newcommand{\R}{\bb{R}}
\newcommand{\C}{\bb{C}}
\newcommand{\N}{\bb{N}}
\newcommand{\M}{\mathbf{M}}
\newcommand{\m}{\mathbf{m}}
\newcommand{\MM}{\mathscr{M}}
\newcommand{\HH}{\mathscr{H}}
\newcommand{\Om}{\Omega}
\newcommand{\Ho}{\in\HH(\Om)}
\newcommand{\bd}{\partial}
\newcommand{\del}{\partial}
\newcommand{\bardel}{\overline\partial}
\newcommand{\textdf}[1]{\textbf{\textsf{#1}}\index{#1}}
\newcommand{\img}{\mathrm{omega}}
\newcommand{\ip}[2]{\left\langle{#1},{#2}\right\rangle}
\newcommand{\inter}[1]{\mathrm{int}{#1}}
\newcommand{\exter}[1]{\mathrm{ext}{#1}}
\newcommand{\cl}[1]{\mathrm{cl}{#1}}
\newcommand{\ds}{\displaystyle}
\newcommand{\vol}{\mathrm{vol}}
\newcommand{\cnt}{\mathrm{ct}}
\newcommand{\osc}{\mathrm{osc}}
\newcommand{\LL}{\mathbf{L}}
\newcommand{\UU}{\mathbf{U}}
\newcommand{\support}{\mathrm{support}}
\newcommand{\AND}{\;\wedge\;}
\newcommand{\OR}{\;\vee\;}
\newcommand{\Oset}{\varnothing}
\newcommand{\st}{\ni}
\newcommand{\wh}{\widehat}

%Pagination stuff.
\setlength{\topmargin}{-.3 in}
\setlength{\oddsidemargin}{0in}
\setlength{\evensidemargin}{0in}
\setlength{\textheight}{9.in}
\setlength{\textwidth}{6.5in}
\pagestyle{empty}

\begin{document}

\begin{center}
{\Large DATA 221 \\  Homework 4  \textbf{(rev 2)}}\\
\textbf{W. Trimble}\\ %You should put your name here
Due: Friday 2022-04-29  - 11:59pm
\end{center}

\vspace{0.2 cm}

Let's look at the MNIST database of handwritten digits.  The data are two sets of 60,000 images (28x28 pixel handwritten digits, encoded as 8-bit greyscale images / 1x784 element vectors.)

To get started, you will need to flatten the images into 1-dimensional vectors and put the class labels into one-hot encoding.

\begin{enumerate}
\item
Display a handful of the images.  Find an average image over at least a thousand training samples; display images for the per-digit mean and the per-digit standard deviation pixel values.

\includegraphics[width=3in]{MNIST.png}

\item
Train neural networks with 0, 1, and more than 1 hidden layers on the 60,000 MNIST training images.  The 0-hidden-layer neural network can be thought of as logistic regression or the single-layer perceptron.
Classify the test set and report the confusion matrix on the test set for each of these networks.

\item
Find and display the images from a handful of cells in the confusion matrix, say, 4 images at each of 4 confusion-matrix-cells.  Which digit pair is hardest to discern? 

% \includegraphics[width=4in]{hastie-generalization.png}
\item 
The first layer of the model can be interpreted as weights-per-pixel; it can be interpreted as a linear inner product filter on the input.   
Take a handful (perhaps 9 or 16) columns from the first-layer weights and display them as images.  Are the input weights for the 4-layer model qualitatively different from the input weights for the 1-layer model?
Hint:  This is an example from the sklearn documentation:

\texttt{https://scikit-learn.org/stable/auto\_examples/neural\_networks/ \
plot\_mnist\_filters.html}

\end{enumerate}
\end{document}


