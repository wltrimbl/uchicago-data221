\documentclass[12pt]{book}

\usepackage{array,epsfig,enumitem}
\usepackage{amsmath}
\usepackage{amsfonts}
\usepackage{amssymb}
\usepackage{amsxtra}
\usepackage{amsthm}
\usepackage{mathrsfs}
\usepackage{color}
\usepackage{eurosym}
\usepackage{times}
%Here I define some theorem styles and shortcut commands for symbols I use often
\theoremstyle{definition}
\newtheorem{defn}{Definition}
\newtheorem{thm}{Theorem}
\newtheorem{cor}{Corollary}
\newtheorem*{rmk}{Remark}
\newtheorem{lem}{Lemma}
\newtheorem*{joke}{Joke}
\newtheorem{ex}{Example}
\newtheorem*{soln}{Solution}
\newtheorem{prop}{Proposition}

\newcommand{\lra}{\longrightarrow}
\newcommand{\ra}{\rightarrow}
\newcommand{\surj}{\twoheadrightarrow}
\newcommand{\graph}{\mathrm{graph}}
\newcommand{\bb}[1]{\mathbb{#1}}
\newcommand{\Z}{\bb{Z}}
\newcommand{\Q}{\bb{Q}}
\newcommand{\R}{\bb{R}}
\newcommand{\C}{\bb{C}}
\newcommand{\N}{\bb{N}}
\newcommand{\M}{\mathbf{M}}
\newcommand{\m}{\mathbf{m}}
\newcommand{\MM}{\mathscr{M}}
\newcommand{\HH}{\mathscr{H}}
\newcommand{\Om}{\Omega}
\newcommand{\Ho}{\in\HH(\Om)}
\newcommand{\bd}{\partial}
\newcommand{\del}{\partial}
\newcommand{\bardel}{\overline\partial}
\newcommand{\textdf}[1]{\textbf{\textsf{#1}}\index{#1}}
\newcommand{\img}{\mathrm{omega}}
\newcommand{\ip}[2]{\left\langle{#1},{#2}\right\rangle}
\newcommand{\inter}[1]{\mathrm{int}{#1}}
\newcommand{\exter}[1]{\mathrm{ext}{#1}}
\newcommand{\cl}[1]{\mathrm{cl}{#1}}
\newcommand{\ds}{\displaystyle}
\newcommand{\vol}{\mathrm{vol}}
\newcommand{\cnt}{\mathrm{ct}}
\newcommand{\osc}{\mathrm{osc}}
\newcommand{\LL}{\mathbf{L}}
\newcommand{\UU}{\mathbf{U}}
\newcommand{\support}{\mathrm{support}}
\newcommand{\AND}{\;\wedge\;}
\newcommand{\OR}{\;\vee\;}
\newcommand{\Oset}{\varnothing}
\newcommand{\st}{\ni}
\newcommand{\wh}{\widehat}

%Pagination stuff.
\setlength{\topmargin}{-.3 in}
\setlength{\oddsidemargin}{0in}
\setlength{\evensidemargin}{0in}
\setlength{\textheight}{9.in}
\setlength{\textwidth}{6.5in}
\pagestyle{empty}

\begin{document}

\begin{center}
{\Large DATA 221 \\  Homework 6  - Embeddings and clsutering}\\
\textbf{W. Trimble}\\ %You should put your name here
Due: Friday 2023-05-19  - 11:59pm
\end{center}

\vspace{0.2 cm}

For the PCA plots of the epicurious dataset:
\begin{enumerate} 

  \item Embedding 
  \begin{enumerate}
    \item Using at least 2000 rows from epicurious, calculate the all-against all distance matrix in Euclidean space for the category labels.  Produce a clustering.
    \item Using the same 2000 rows, calculate the all-against all distance matrix with a Minkowski metric for the category labels.  Produce a clustering.
    \item Map the category labels to a \texttt{word2vec} language embedding, calculate bag-of-words vectors for each recipe, and produce a clustering.   Visualize the clustering (PCA, TSNE, or clustermap would do)
    \item Compare the three clusterings visually; make three PCA or TSNE scatterplots and label them suitably.
  \end{enumerate}

\item  Feature selection / embedding

The Online News Popularity dataset collected by Fernandes et al. 

 \texttt{https://archive.ics.uci.edu/ml/datasets/online+news+popularity} 

contains 58 features extracted from 39,000 webpages and one response variable number of shares.  

\begin{enumerate}
\item  
Extract the text from the 39,000 webpages and check whether you can reproduce the "text length" field.
\item 
Calculate bag-of-words word2vec feature vectors for each of the articles and perform clustering (kmeans or hierarchical will do)
\item 
Plot the within-cluster-variance vs. cluster number graph and choose a number of clusters.
\item 
See if you can identify the clusters.  You can try to use the vectors or you can inspect the cluster members. 

\end{enumerate}

\end{enumerate}
\end{document}


